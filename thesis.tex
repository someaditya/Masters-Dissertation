%%%%%%%%%%%%%%%%%%%%%%%%%%%%%%%%%%%%%%%%%%%%%%%%%%%%%%%%%%%%%%%%%%%%%%%%%%%%%
%%%
%%% File: utthesis2.doc, version 2.0jab, February 2002
%%%
%%% Based on: utthesis.doc, version 2.0, January 1995
%%% =============================================
%%% Copyright (c) 1995 by Dinesh Das.  All rights reserved.
%%% This file is free and can be modified or distributed as long as
%%% you meet the following conditions:
%%%
%%% (1) This copyright notice is kept intact on all modified copies.
%%% (2) If you modify this file, you MUST NOT use the original file name.
%%%
%%% This file contains a template that can be used with the package
%%% utthesis.sty and LaTeX2e to produce a thesis that meets the requirements
%%% of the Graduate School of The University of Texas at Austin.
%%%
%%% All of the commands defined by utthesis.sty have default values (see
%%% the file utthesis.sty for these values).  Thus, theoretically, you
%%% don't need to define values for any of them; you can run this file
%%% through LaTeX2e and produce an acceptable thesis, without any text.
%%% However, you probably want to set at least some of the macros (like
%%% \thesisauthor).  In that case, replace "..." with appropriate values,
%%% and uncomment the line (by removing the leading %'s).
%%%
%%%%%%%%%%%%%%%%%%%%%%%%%%%%%%%%%%%%%%%%%%%%%%%%%%%%%%%%%%%%%%%%%%%%%%%%%%%%%

%%%%%%%%%%%%%%%%%%%%%%%%%%%%%%%%%%%%%%%%%%%%%%%%%%%%%%%%%%%%%%%%%%%%%%%%%%%%%
%%%
%%
%% This file, and the corresponding tcdthesis.sty the accompanied it, have
%% been modified for the M.Sc. styles used in Trinity College, Dublin
%%
%%
%%%%%%%%%%%%%%%%%%%%%%%%%%%%%%%%%%%%%%%%%%%%%%%%%%%%%%%%%%%%%%%%%%%%%%%%%%%%%
\documentclass[a4paper, 12pt, oneside]{report}         %% LaTeX2e document.
\usepackage {tcdthesis}              %% Preamble.
\usepackage{graphicx}
\usepackage[utf8]{inputenc}
\usepackage{amsmath}
\usepackage{natbib}
\usepackage{hyperref}



\mastersthesis                     %% Uncomment one of these; if you don't
% \phdthesis                         %% use either, the default is \phdthesis.

\thesisdraft                       %% Uncomment this if you want a draft
                                     %% version; this will print a timestamp
                                     %% on each page of your thesis.

\leftchapter                       %% Uncomment one of these if you want
%\centerchapter                      %% left-justified, centered or
% \rightchapter                      %% right-justified chapter headings.
                                     %% Chapter headings includes the
                                     %% Contents, Acknowledgments, Lists
                                     %% of Tables and Figures and the Vita.
                                     %% The default is \centerchapter.

% \singlespace                       %% Uncomment one of these if you want
\oneandhalfspace                   %% single-spacing, space-and-a-half
% \doublespace                       %% or double-spacing; the default is
                                     %% \oneandhalfspace, which is the
                                     %% minimum spacing accepted by the
                                     %% Graduate School.

\renewcommand{\thesisauthor}{Some Aditya Mandal}         %% Your official TCD name.
\renewcommand{\thesismonth}{August}                  %% Your month of graduation.
\renewcommand{\thesisyear}{2018}                      %% Your year of graduation.
\renewcommand{\thesistitle}{Neural Machine Translation for English to Hindi in Tourism Domain}            %% The title of your thesis; use mixed-case.
\renewcommand{\thesisauthorpreviousdegrees}{ , B.Eng}  %% Your previous degrees, abbreviated; separate multiple degrees by commas.
\renewcommand{\thesissupervisor}{Dr. Kevin Koidl}      %% Your thesis supervisor; use mixed-case and don't use any titles or degrees.
% \renewcommand{\thesiscosupervisor}{}                %% Your PhD. thesis co-supervisor; if any.

% \renewcommand{\thesiscommitteemembera}{}
% \renewcommand{\thesiscommitteememberb}{}
% \renewcommand{\thesiscommitteememberc}{}
% \renewcommand{\thesiscommitteememberd}{}
% \renewcommand{\thesiscommitteemembere}{}
% \renewcommand{\thesiscommitteememberf}{}
% \renewcommand{\thesiscommitteememberg}{}
% \renewcommand{\thesiscommitteememberh}{}
% \renewcommand{\thesiscommitteememberi}{}


\renewcommand{\thesisauthoraddress}{...}

\renewcommand{\thesisdedication}{...}     %% Your dedication, if you have one; use "\\" for linebreaks.


%%%%%%%%%%%%%%%%%%%%%%%%%%%%%%%%%%%%%%%%%%%%%%%%%%%%%%%%%%%%%%%%%%%%%%%%%%%%%
%%%
%%% The following commands are all optional, but useful if your requirements
%%% are different from the default values in tcdthesis.sty.  To use them,
%%% simply uncomment (remove the leading %) the line(s).

\renewcommand{\thesisdegree}{Master of Science in Computer Science}  
                                     %% default is "DOCTOR OF PHILOSOPHY"
                                     %% for \phdthesis or "MASTER OF ARTS"
                                     %% for \mastersthesis.  Provide the
                                     %% correct FULL OFFICIAL name of
                                     %% the degree.
\renewcommand{\thesisdegreestream}{ (Intelligent Systems)}
                                     %% Default is empty. This is used on
                                     %% the title page of the thesis.

\renewcommand{\thesisdegreeabbreviation}{M.Sc.}
                                     %% Use this if you also use the above
                                     %% command; provide the OFFICIAL
                                     %% abbreviation of your thesis degree.
\renewcommand{\thesistype}{Dissertation}    %% Use this ONLY if your thesis type
                                     %% is NOT "Thesis" for \phdthesis
                                     %% or \mastersthesis.
                                     %% Provide the OFFICIAL type of the
                                     %% thesis; use mixed-case.

% \renewcommand{\thesistypist}{...}  %% Use this to specify the name of
                                     %% the thesis typist if it is anything
                                     %% other than "the author".

%%%
%%%%%%%%%%%%%%%%%%%%%%%%%%%%%%%%%%%%%%%%%%%%%%%%%%%%%%%%%%%%%%%%%%%%%%%%%%%%%


\begin{document}                                  %% BEGIN THE DOCUMENT

\thesistitlepage                                  %% Generate the title page.

\thesisdeclarationpage                %% Generate the declaration page.

\thesispermissionpage                 %% Generate the copyright permission page

%\thesisdedicationpage                             %% Generate the dedication page.

\begin{thesisacknowledgments}                     %% Use this to write your
%% acknowledgments; it can be anything
I would like to extend my sincerest thanks and appreciation to Dr. Kevin Koidl for his continuous motivation and immense guidance throughout the phase of research. Kevin consistently allowed this study to be much of my own work, encouraging my ideas and development throughout the research, while always helping me in spots of difficulty, and steering me in the right direction whenever needed.

To my Dad, who’s my biggest inspiration in life, I must express my greatest gratitude for providing me with everything I could have ever needed, and his continuous encouragement throughout my years of study.This accomplishment would not have been possible without my Dad. Thank you Dad.

To my Mom, who's not here to witness this moment.I will be grateful to her forever, for her influence on my life, I owe my entire education to her, and am sorry that she has not lived to see me graduate. Thank you Mom.
\end{thesisacknowledgments}                      %% allowed in LaTeX2e par-mode.

\begin{thesisabstract}                          %% the abstract for your thesis
Hindi is one of the major languages of the world, which is spoken by over 700 million people around the world. In India, more than 550 million people speak Hindi, whereas the English speakers are limited to around 12\%. In this modern digital age, most of the digital content that is produced in India and the rest of the World is predominately in English. This leaves a considerable amount of Indian population to be digitally neglected as there are limited digital content in their native language or the most widely used language Hindi. This results in an immense potential for the English to Hindi neural machine translation by enabling automatic translations for millions of existing digital English articles. The recent development of deep neural networks enabled machine translation to create translations that are close to human generated translations. There is an immense opportunity to explore the English to Hindi Neural Machine Translation and the domain adaptation of neural translation model by generating domain specific translations. In this research, the tourism domain was chosen based on considerable lack of translated content in this domain. To overcome this shortcoming this work creates a English-Hindi Neural Machine Translation Model which generate above average translations. Furthermore, domain Specific Fine-tuning of the pre-trained model is conducted In order to extend the domain adaptability of the model. All models are tested and evaluated on domain specific data using the BLEU scores. This dissertation report presents the fundamentals of neural machine translation and draws conclusions about their domain specific adaptability.



\end{thesisabstract}

\begin{thesissummary}                           %% The summary page for your thesis

This work targets the English to Hindi machine translation using the Neural Machine Translation Models. It is part of the ongoing efforts within Neural Machine Translation research, where studies strive to understand the use of Deep Neural Networks to create automatic machine translations which are on par with human translations.

This research is focused on improving the quality of English to Hindi translations by applying a novel approach to training and using Neural Machine Translation Models. For this a gap in the State of the Art was discussed leading to the conclusion that there hasn't been much focus on English to Hindi translation. To overcome this gap, this research considers the huge Hindi speaking population of India and the lack of digital content in Hindi, and thus points to a need for English to Hindi neural machine translation.

This research establishes that the monolingual corpora from the target language can be utilized for low resource language pairs to influence the quality of translations. This work uses a small amount Hindi monolingual data to generate synthetic English sentences, which are then utilized for the fine-tuning of the baseline translation model. The presented novel approach  can be applied in the adaptation of the back-translation method for different low-resource language pairs. The research illustrates that domain specific training of a neural translation model with cross-domain specific training data leads to domain adaptation. The domain specific fine-tuning data, which consisted transcripts from TED Talks, was used to fine-tune the translation model. It was tested with the test data consisting of sentences extracted from travel blogs. The results of the experiments indicated that a cross-domain trained neural translation model performs well on the in-domain data, and subsequently generates above average quality translations. In a broader perspective this research bridges the identified gap, of a lacking English to Hindi focus, in the State of Art of Machine Translation by creating a new baseline translation model for large English-Hindi corpus leading to better than average results.
	
\end{thesissummary}


\tableofcontents                                  %% Generate table of contents.
\listoftables                                     %% Uncomment this to generate list of tables.
\listoffigures                                    %% Uncomment this to generate list of figures.

%%
%% Include thesis chapters here...
%%
  
\chapter{Introduction}
In this chapter, the motivation driving the research is introduced. The research question, the objectives for the research are identified along with the potential research challenges that may emerge. At the last, a technical approach is proposed for the research and a dissertation outline is provided which gives a view of the overall structure of the document.
\section{Motivation}
Hindi is one of the major languages of the world, which is mostly used in the Indian subcontinent, and is perceived as a local language in Mauritius, Trinidad and Tobago, Guyana, and Suriname. Also, it has a official language status in India and it is the most widely used language in India. As indicated by the 2001 Census of India\footnote{\url{http://www.censusindia.gov.in/Census_Data_2001/India_at_glance/glance.aspx},accessed:25.08.2018}, Hindi has 422 million native speakers and more than 500 million aggregate speakers (\citeauthor{ wiki:hindi}, \citeyear{ wiki:hindi}). It is additionally an official language of the Union Government of India along with English and also has significant importance in Indian states like Uttar Pradesh, Bihar, Rajasthan, Himachal Pradesh and so forth. Numerous languages and dialects in the Gangetic plains of India are closely associated with Hindi e.g.Bhojpuri, Awadhi, Maithili,etc. Hindi is also spoken by a large number of Madhesis from Nepal. The official language of Pakistan, Urdu is mutually intelligible with Hindi apart from a dedicated vocabulary.Hindi is also the fourth-most spoken-language on the planet, third-most spoken language alongside Urdu (both are registers of the Hindustani language) which is around 700 million total speakers \footnote{\url{https://www.babbel.com/en/magazine/the-10-most-spoken-languages-in-the-world/},    accessed:25.08.2018}. Interestingly, English is spoken by just around 125 million individuals  in India which around 12.18\% of the total population(\citeauthor{ wiki:english}, \citeyear{ wiki:english}).  


The most of the digital content that is produced in India and the rest of the World is predominately in English\footnote{\url{https://www.internetworldstats.com/stats7.htm/}, accessed:25.08.2018}.This leaves a considerable amount of Indian population to be digitally neglected as there are limited digital content in their native language or the most widely used language Hindi. So, there is an immense potential for the English to Hindi machine translation which  can enable automatic translations for millions of existing digital content which are in English. 

A lot of researches have been done in the field of neural machine translation and the results have shown that neural machine translation models can generate humanly translations ( \citeauthor{NIPS2014_5346} \citeyear{NIPS2014_5346}; \citeauthor{DBLP:journals/corr/BahdanauCB14} \citeyear{DBLP:journals/corr/BahdanauCB14}; \citeauthor{45610} \citeyear{45610}). Though a very small number of researches were pursued for studying the use of neural machine translation for English to Hindi translations and other Indic\footnote{relating to or denoting the group of Indo-European languages comprising Sanskrit and the modern Indian languages which are its descendants. {\url{https://www.merriam-webster.com/dictionary/Indic}}} Language pairs. 

There is a huge gap in State of Art related to the use of neural machine translation for English to Hindi translation. There exists an exciting opportunity to design new neural machine translation models to create above average translations in English to Hindi using a large corpora. There is also an immense opportunity to explore the domain adaptation of neural translation model to generate domain specific translations. For this research, the tourism domain was selected due to the observation that there is considerable lack of digital content in the domain. Though a lot of travel related digital content is published daily on the internet, but very few are available in Hindi. The availability of travel content in Hindi can influence a major portion of population in making better travel choices , and help to boost the Tourism industry which forms around 9.4\% of India's economy.This immediate potential, characteristic to the domain adaption of the neural translation model, motivates the research question specific to this work of dissertation (\citeauthor{wiki:tourism} , \citeyear{wiki:tourism}).

\section{Research Question}
This dissertation explores the question how cross-domain specific training data can help the  neural machine translation model to generate above average translations for domain specific data.

\section{Research Objectives}
To address this research question, five specific research objectives have been defined:
\begin{enumerate}
    \item To create a baseline neural translation model for English to Hindi translation which has the ability to generate above average quality translations by using a large parallel corpus.
    \item To create a new Domain Specific Corpus for the Tourism Domain.
    \item To Fine Tune (Re-Train) the neural translation model by utilizing  a  Hindi Monolingual Corpus.
    \item To Fine Tune (Re-Train) the neural translation model with Cross-Domain Specific Training Data.
    \item To Test and Evaluate the neural translation model with Domain Specific Test Data from the Domain Specific Corpus.
\end{enumerate}
\section{Research Challenges}

Along with the research objective mentioned earlier, there exists various significant research challenges which are specific to this area of neural machine translation. These research challenges must be addressed for this research to be conducted successfully. 
\begin{enumerate}
    \item Neural Machine Translation is a vast research area, and the study is extensive. Hindi is a complex language which follows different word orders and different sentence formation methods. There can be different translations in Hindi for a given sentence in English. Creating an above average English-Hindi translations involve in-depth knowledge of the Hindi language and the theory behind neural machine translation. 
    
    This work takes into consideration the diversity of the Hindi language and the neural translation system is designed to address this issue. 
    \item Neural Translation models need large corpus for the purpose of training. The selection of corpus thus plays an important role while creating an above average neural translation model. This work will utilize the IIT-Bombay English-Hindi parallel corpus which is the largest publicly available English-Hindi corpus. 
    \item The utilization of monolingual corpora in neural machine translation is a challenging task. As, overuse of synthetic data may deteriorate the translation quality and deviate the research from the main objective. To address this, a small number of monolingual corpus will be used for improving the translation quality.
    \item Domain adaptation of the neural translation model is a tricky piece of work. If a neural translation model is trained with huge amount of new data, which totally unrelated to the original training data,it may result in bad translations. So, for this work, a closely related corpus is created for the domain specific training from the TED Talks website\footnote{\url{https://www.ted.com/}, accessed:17.08.2018}. 
\end{enumerate}

\section{Overview of Technical Approach}

To address the research objectives outlined in the previous section, a large parallel corpus is selected and a English to Hindi neural translation model is implemented using an open-source neural machine translation toolkit. The domain specific corpus is generated by scrapping a publicly available website and processing the text using an open source natural language processing libraries. Further, a Hindi monolingual corpus is used to generate synthetic English sentences, and the newly created sentence pairs are used to re-train the base line translation model. The domain adaptation process involves, the re-training of the neural translation model using the domain specific corpus.  The models are tested and evaluated using the most commonly used metric BLEU on the domain specific test data. 

\section{Dissertation Outline}
\begin{itemize}

\item Chapter 1 provides an introduction to the project. It presents a brief background
and motivation behind the project and then presents the research questions addressed
in this dissertation.

\item Chapter 2 presents the State of the Art in the Machine Translation. This chapter briefly introduces the technologies and research areas involved in this project and then reviews the relevant literature.

\item Chapter 3 provides a detailed description of the design decisions for the research, building on the foundations and stemming from the related work reviewed in Chapter Two.

\item  Chapter 4 discusses the implementation details of this research. The technologies that were used are described, alongside the justification for selecting them.

\item Chapter 5 provides an overview and discussion of the observations gathered throughout
the research. The results of the research are evaluated against the research objectives set out in Chapter One.

\item Chapter 6 concludes the dissertation in which conclusions are drawn about the research and highlights the research contribution.It ends with a discussion on the future work prospects.
\end{itemize}


  \chapter{State of Art}
In this chapter, the background for the research is investigated. To further elucidate the motivation behind the research, the potential implications of formally measuring rumour impact are discussed in detail.

\section{Background}

\section{Statistical Machine Translation}
\subsection{Word Based Translation Systems}
\subsection{Phrase Based Translation Systems}
A novel phrase-based translation model and decoding algorithm was proposed by Koehn et al (2022) which enabled them to evaluate and compare several previously proposed phrase-based translation models. They designed a uniform framework to compare different other translation models. The model proposed by Koehn et al (2022) was based on the noisy channel model and they used the Bayes rule to reformulate the translation probability for translating a foreign sentence $f$ into English $e$ as

$$
argmax_ep(e|f)= argmax_ep(f|e)p(e)
$$

That allowed for language model $p(e)$ and a separate translation model $p(f|e)$.

During the decoding phase, the foreign input sentence $f$ was segmented into a sequence of $I$ phrases $\bar{f_1^1}$ . Uniform probability distribution over all possible segmentation were assumed by the authors. Each foreign phrase $\bar{f_i}$ in $\bar{f_1^1}$ was translated into an English phrase $\bar{e_i}$, though there were reordering of the English sentences. A probability distribution $\phi(\bar{f_i}|\bar{e_i})$ modeled the entire phrase translation and due to the Bayes Rule, the direction of translation is inverted from a modeling standpoint. 

Reordering of the English output phrases was modeled by a relative distortion probability distribution $d(a_i$ - $b_{i-1})$, where the start position of the foreign phrase that was translated to the English phrase was denoted by $a_i$ and the end position of the foreign phrase was denoted by $b_{i-1}$. Throughout their experiments, they trained the distortion probability distribution $d (.)$ using a joint probability model. Further to optimize the performance of the model , they introduced a factor $w$ for each generated English word in addition to trigram language model $p_{LM}$. 

The best English output sentence $e_{best}$ given a foreign input sentence $f$ according to their model is ,

$$
 e_{best}= argmax_ep(e|f)
         = argmax_ep(e|f)
$$




The phrase-based decoder developed by Koehn et al. (2022) employed a beam search algorithm similar to the one by Jelinek (1998)  for the comparison of different phrase-based translation models.
\subsection{Syntax Based Translation}
A syntax-based statistical translation model was proposed by Yamada et al (2001). Their model transformed a source-language parse tree into a target-language string by applying stochastic operations at each node. Those operations captured the linguistic differences such as word order and case marking. The model produced word alignments which were better those produced by IBM Model 5. [Reference]

They assumed that an English parse three $\varepsilon$  is transformed into a French sentence $f$. The English parse tree $\varepsilon$ consisted of nodes $\varepsilon_1$,$\varepsilon_2$,.. $\varepsilon_n$ and the French sentence consisted of French words $f_1,f_2,..,f_m$.
The authors considered three random variables, \textbf{N}, \textbf{R}, and \textbf{T} as channel operations which were applied to each node. \textit{Insertion} \textbf{N} is an operation that inserts a French word just before or after the node and it can be either none, left or right. \textit{Reorder} \textbf{R} is an operation that is used to change the order of the children of the node. \textit{Translation} \textbf{T} is an operation which translates a terminal English leaf word into a French word. 
\begin{figure}
\includegraphics[width=\textwidth,,height=250pt]{figures/syntax1.png}
\caption{: Viterbi Alignments: Yamata et al's model (left) and IBM Model 5 (right). Darker lines are judged more
correct by humans.} \label{fig1}
\end{figure}

\begin{center}
\begin{tabular}{ |c|c|c| } 
 \hline & Alignment ave. score & Perfect sents \\ 
 \hline Yamata et al Model & 0.582 & 10 \\ 
 \hline IBM Model 5 & 0.431 & 0 \\ 
 \hline
\end{tabular}
\end{center}

\subsection{Moses and Factored Translation Model}
Moses is an open-source implementation of the statistical machine translation developed by Koehn et al (2007). The statistical machine translate systems are trained on huge quantities of parallel bilingual data and larger quantities of monolingual data. The parallel data is collection of sentences in two different languages which is sentence aligned, that is the sentence in one language is matched with sentence in corresponding language. In Moses, the system takes parallel data for the training process and uses the co-occurrences of phrases to infer translation correspondences between two languages of interest. In phrase-based machine translation, these correspondences are essentially between ceaseless arrangements of words, whereas in hierarchical phrase-based machine translation or syntax-based translation, more structure is added to the correspondences. Apart from being an open-source toolkit for SMT, Moses extended the phrase-based translation with factors and confusion network decoding. The phrase-based model in statistical machine translation was limited to the mapping of small text chunks with no express utilization of etymological data, be it morphological, syntactic, 
or on the other hand semantic. Previous researches showed that these additional sources of information are valuable when integrated into pre-processing or post-processing steps. Moses also integrated confusion network decoding, a mechanism which allowed translation of ambiguous input and enabled tighter integration of speech recognition and machine translation. The machine translation system examines a network of different word choices instead of passing along the one-best output of the recognizer. 
\subsubsection{Factored Translation Model}
The non-factored SMT such as phrase-based SMT typically dealt only with surface form of words and had one phrase table as shown in Figure 1. 
\begin{figure}
\includegraphics[width=\textwidth,,height=250pt]{figures/moses1.png}
\caption{Non-factored Translation} \label{fig1}
\end{figure}

In factored translation model, the surface forms may be augmented with different factors, such as POS tags or lemma. This creates a factored representation of each word, Figure 2 [Refer].
\begin{figure}
\includegraphics[width=\textwidth,,height=250pt]{figures/moses2.png}
\caption{Factored Translation} \label{fig1}
\end{figure}

The authors suggested that mapping of source phrases to target phrases might be decomposed into a few stages. Decomposition of the decoding process into different steps implied that diverse components can be modelled independently. Modelling factors in isolation takes into consideration adaptability in their application. It can likewise increment accuracy and decrease sparsity by limiting the number conditions for each step. For example, a surface form can be decomposed to surface forms and lemma, as shown in figure3.
\begin{figure}
\begin{center}
\includegraphics[width=300pt]{figures/moses3.png}
\caption{Example of graph of decoding systems} \label{fig1}
\end{center}
\end{figure}

The graph was allowed to be user definable, thus it provided a scope for experimentation with different configurations to find the optimum configuration for the given language pair and data. The authors considered the factors on the source sentence to be fixed, therefore they did not implement any decoding step to create source factors from other source factors. They designed Moses such as every factor on the target language could have its own language model.  Many factors such as lemmas and POS tags are sparser than surface forms so it was possible to create a higher order language models for these factors.

\subsubsection{Confusion Network Decoding}
The authors wanted to meet the increasing demands of integrating machine translation technology into bigger information processing systems with upstream NLP/speech processing tools such as named entity recognizers, speech recognizers, morphological analyzers, etc. The authors experimented with confusion networks by focusing on the speech translation case, where initially the input is generated by a speech recognition system. Their immediate goal was to improve the language translation by combining the speech recognition and machine translation models. The biggest problem in translation of spoken language was, its proneness to speech recognition errors which used to corrupt the input syntax and its meaning. Previous researches also showed that better translations can be obtained from the transcriptions of the speech recognizer.  The authors found that significant improvements could have been achieved by applying machine translation techniques on larger sets of transcription texts generated by the speech recognizers and combining the scores of acoustic models, language models, and translation models. 

In Moses, they implemented the confusing network decoding as discussed in (Bertoldi and Federico 2005), and they developed a simpler translation model and a more efficient implementation of the search algorithm.

\subsection{Challenges in SMT}

Statistical Machine Translation generated translation using statistical models whose parameters are derived from the analysis of bilingual text corpora. 

There were several challenges faced in statistical machine translation, which are as follows:

\begin{itemize}
    \item \textbf{Lack of Large Parallel Corpora}   Statistical Machine Translation systems need large sets of parallel data for the translation task. But the unavailability of large corpora in low resource language posed a challenge to the SMT and it affected the efficiency of translation models in several low resource languages.
    \item\textbf{Sentence Alignment }   In parallel corpora, a single sentence in one language when translated into other language may be more than one sentence, thus aligning such sentences in parallel becomes a challenge. Further, algorithms such as Gale-Church alignment algorithm needed to be used for sentence alignment.
     \item\textbf{Word Alignment }      Most of the popular parallel corpora is sentence aligned or sentence alignment can be done using the previously mentioned Gale-Church Algorithm. The real challenge lies for word alignment, to know which word in source language aligns with which word in the target language, though IBM models (Reference) and HMM-approach (Reference) provides a better solution to this problem.
     \item\textbf{Statistical anomalies }  Sometimes the real-world training sets tends to overwrite the translations of proper nouns. An abundance of particular noun in the training set may tend overwrite a less frequent noun in translated sentence. An example would be like,"\textit{Thóg mé an traein go Gaillimh} " in Irish should ideally translate to "\textit{I took the train to Galway}" in English, but gets to "\textit{I took the train to Cork"} due to abundance of "\textit{train to Cork}" in the training data.
     \item\textbf{Idioms }Depending upon the corpora utilized for the translation task, idioms may not translate "idiomatically". For instance, utilizing Canadian Hansard [Reference] as the bilingual corpus, "hear" may constantly be meant "Bravo!" since in Parliament "Hear, Hear!" moves toward becoming "Bravo!". 
     \item\textbf{Different Word Orders}    Different languages follow different word orders. In some languages, classification can be done by naming the typical order of subject (S), verb (V) and Object (O) in a sentence and in some languages, it can be represented as SVO or VSO. An example, the English sentence \textit{“Tom is eating vegetables”} is having the SVO word order, where as its translation in Hindi can be either \textit{“Tom vegetables kha raha hai”} which is SOV or \textit{“Vegetables kha raha hai Tom”} which is of the OVS order. 
     \item\textbf{Out of vocabulary (OOV) words}    The SMT systems typically store the different word forms as separate symbols which are unrelated to each other and words forms or phrases that were not in the training data which could not be translated. This is generally due to the lack of training data, changes in the human domain or some morphological differences. 
     \item\textbf{Mobile devices}   The rapid increase in the processing power of tablets and cell phones, combined with the wide accessibility of machine translation systems. However, combining other Speech recognition systems with SMT on mobile devices raises problems of sentence segmentation, de-normalization and punctuation prediction which is a basic necessity for quality translations.
\end{itemize}
\section{Neural Machine Translation Systems}
\subsection{Introduction}
\subsection{Google's Neural Machine Translation}
They proposed a model (see Figure 1) which follows the common sequence-to-sequence learning framework [41] with attention [2]. The model has three components: an encoder network, a decoder network and an attention network. The encoder network converts a source sentence into a list of vectors, one vector per input symbol. When this list of vectors is passed to decoder network it produces one symbol at a time until it encounters the special end-of-sentence symbol (EOS). The encoder and decoder are connected through an attention module which allows the decoder to focus on different regions of the source sentence during the course of decoding. 
\begin{figure}
\includegraphics[width=\textwidth]{figures/gnmt1.png}
\caption{ The model architecture of GNMT, Google’s Neural Machine Translation system. On the left
is the encoder network, on the right is the decoder network, in the middle is the attention module. The
bottom encoder layer is bi-directional: the pink nodes gather information from left to right while the green
nodes gather information from right to left. The other layers of the encoder are uni-directional. Residual
connections start from the layer third from the bottom in the encoder and decoder. The model is partitioned
into multiple GPUs to speed up training. In our setup, we have 8 encoder LSTM layers (1 bi-directional layer
and 7 uni-directional layers), and 8 decoder layers. With this setting, one model replica is partitioned 8-ways
and is placed on 8 different GPUs typically belonging to one host machine. During training, the bottom
bi-directional encoder layers compute in parallel first. Once both finish, the uni-directional encoder layers
can start computing, each on a separate GPU. To retain as much parallelism as possible during running
the decoder layers, we use the bottom decoder layer output only for obtaining recurrent attention context,
which is sent directly to all the remaining decoder layers. The softmax layer is also partitioned and placed on
multiple GPUs. Depending on the output vocabulary size we either have them run on the same GPUs as the
encoder and decoder networks, or have them run on a separate set of dedicated GPUs} \label{fig1}
\end{figure}


For Notation, they used bold lower case to denote the vectors, bold upper case to denote the matrices, cursive upper case to denote the sets, capital letters to denote the sequences and lower case to denote the individual symbols in a sequence. 

They assumed (X,Y) as a source and target sentence pair , and $X$ = $x_1,x_2,x_3,..,x_M$ as the sequence of M symbols in the source text and $Y$= $y_1,y_2,y_3,.., y_N$ as the sequence of $N$ symbols in target text. The encoder is a simple function of the following form:
$$x_1,x_2,..,x_M = EncoderRNN(x_1,x_2,..,x_M)$$

Their decoder network is implemented as an amalgamation of an RNN network and a softmax Layer. The decoder RNN network creates a hidden state $Y_i$ for the next symbol to be predicted, which then goes through the softmax layer to generate a probability distribution over candidate output symbols. In their experiments the authors found out that Neural Machine Translation systems must have deep RNNs for encoder and decoder networks to achieve good accuracy, they need to capture the minute irregularities in the source and target languages. 

The Attention Model they implemented in their research is similar to [2]. More precisely, they assumed yi-1 to be the decoder RNN from the previous decoding step which is the bottom decoder layer. Attention context aI , for the current time step is computed according to the following formulas:



St



Pt


Ai

Where AttentionFunction in their implementation is a feed forward network with one hidden layer.

The authors acknowledged the fact that simply tacking up more Layers of LSTM makes the network slower and difficult to train, likely due to exploding and vanishing gradient problems [33,22]. The simple stacked LSTM layers work well up to 4 to 6 layers but performed very poorly beyond 8 layers.

\begin{figure}
\includegraphics[width=\textwidth]{figures/gnmt2.png}
\caption{The difference between normal stacked LSTM and our stacked LSTM with residual connections.
On the left: simple stacked LSTM layers [41]. On the right: our implementation of stacked LSTM layers
with residual connections. With residual connections, input to the bottom LSTM layer (x
0
i
’s to LSTM1) is
element-wise added to the output from the bottom layer (x
1
i
’s). This sum is then fed to the top LSTM layer
(LSTM2) as the new input.} \label{fig1}
\end{figure}

The authors were motivated by the idea of modelling differences between an intermediate layer’s output and the targets, which has shown to work well for many projects in the past [16,21,40], they introduced residual connections among the LSTM layers in a stack (See Figure 2). Residual Connection greatly improved the gradient flow in the backward pass, which allowed them to train their encoder and decoder networks with 8 LSTM layers. In translation systems, the information required to translate some words in the output language can appear anywhere in the source language. They decided to use a bi-directional RNN for the encoder to have the optimum scenario in the encoder network. They used the bi-directional connections for the bottom encoder layer while keeping other layer uni-directional to allow maximum parallelization during computation. 

\begin{figure}
\includegraphics[width=\textwidth]{figures/gnmt3.png}
\caption{The structure of bi-directional connections in the first layer of the encoder. LSTM layer $LSTM_f$
processes information from left to right, while LSTM layer $LSTM_b$ processes information from right to left.
Output from $LSTM_f$ and $LSTM_b$ are first concatenated and then fed to the next LSTM layer $LSTM_1$.} \label{fig1}
\end{figure}

Due to the complexity of their model the authors implemented model parallelism and data parallelism to speed up the training. But Model Parallelism came up with certain constraints on their Model Architecture like they could not bi-directional LSTM layers for their all encoder layers.  The authors implemented the Wordpiece model which was initially developed by Google Speech Recognition System [35] for solving Japanese/ Korean segmentation problem to solve the out-of-vocabulary translation problems. 

\begin{figure}
\includegraphics[width=\textwidth]{figures/gnmt4.png}
\caption{Histogram of side-by-side scores on 500 sampled sentences from Wikipedia and news websites for a
typical language pair, here English to Spanish (PBMT blue, GNMT red, Human orange). It can be seen that
there is a wide distribution in scores, even for the human translation when rated by other humans, which
shows how ambiguous the task is. It is clear that GNMT is much more accurate than PBMT.} \label{fig1}
\end{figure}

\subsection{OpenNMT}
OpenNMT is an open source framework for Neural Machine Translation developed by Harvard University and SYSTRAN. They designed OpenNMT with three aims: a) prioritize first training and test efficiency b) maintain model modularity and readability, c) support significant research extensibility. [Main OpenNMT] . 

They designed and developed  OpenNMT at Harvard, as a successor to $seq2seq-atn$, it was completely rewritten for ease of efficiency ,readability and generalizability. The OpenNMT framework includes the Vanilla NMT Models as well as provides support for attention, gating, stacking, input feeding, regularization, beam search and other options needed for State-of-the-art performance. They implemented the main system in the Lua/Torch mathematical framework which can be easily extended using Torch's internal neural network components. 

The authors designed the OpenNMT to meet certain goals: System Efficiency, code modularity and model extensibility. 

\begin{figure}
\includegraphics[width=\textwidth]{figures/openmt.png}
\caption{ Schematic view of neural machine translation. The red source words are first mapped to
word vectors and then fed into a recurrent neural network (RNN). Upon seeing the heosi symbol, the final
time step initializes a target blue RNN. At each target time step, attention is applied over the source RNN
and combined with the current hidden state to produce a prediction p(wt|w1:t−1, x) of the next word. This
prediction is then fed back into the target RNN} \label{fig1}
\end{figure}


\subsubsection{System Efficiency}

One of the biggest concerns about the Neural Machine Translation Systems is the training efficiency, the systems take huge time to train sometimes ranging from days to weeks. The authors tried to address the issue and tried to make the training slightly faster in OpenNMT. 

\textbf{Memory Sharing} The most common reason for high time consumption while training GPU-based NMT models is due to the memory size restrictions which limit the batch size. Though neural network toolkits like Torch has been designed to trade-off the surplus memory allocations for speed and declarative simplicity. For OpenNMT they implemented an external memory sharing system that utilizes the time-series control flow of NMT systems and vigorously shares the internal buffers between the clones. This implementation of vigorously memory reuse results in conservation of about 70\% of GPU memory with the default model size.

\textbf{Multi-GPU} The authors added support for multi-GPU training using data parallelism. Two modes were made available: synchronous and asynchronous training. In synchronous training, the batches run concurrently on parallel GPU and gradients aggregated to update master parameters before resynchronization on each GPU for the next batch. While in asynchronous training, the batches run independent on each GPU and independent gradients accumulated to the master copy of the parameters. Asynchronous SGD is known to provide faster convergence [Reference e (Dean et al., 2012). ] 

\textbf{C/Mobile/GPU Translation} The authors included several different translation models for different run-time environments in OpenNMT: a batched CPU/GPU implementation for quick translation of large set of sentences, a simple single-instance implementation for use on mobile devices and a specialized C implementation. 

\subsubsection{Modularity for Research}

The secondary goal for the authors were to make the code readable for non-experts and they targeted this goal by explicitly separating out optimizations from the core model and by including tutorial documentations within the code. 

Need to write here 

\textbf{Extensibility}

The field of Deep Learning is quickly evolving and technologies change very frequently. The authors went to perform different case studies to ensure that OpenNMT adhere to code extensibility and support the future variants.

Need to write a little more


\begin{table}[h!]
\caption{Performance Results for EN→DE on WMT15
tested on newstest2014. Both system 2x500 RNN, embedding
size 300, 13 epochs, batch size 64, beam size 5. We
compare on a 50k vocabulary and a 32k BPE setting. OpenNMT
showed improvements in speed and accuracy compared
to Nematus.\label{long}}
\centering
 \begin{tabular}{ |c|c|c|c|c| } 
 \hline Vocab & System & Speed Train & tok/sec Trans & BLEU \\ 
 \hline V=50K & Nematus & 3393 & 284 & 17.28 
 &&OpenNMT&4185&380&17.60\\ 
  \hline V=32K & Nematus & 3221 & 252 & 18.25 
 &&OpenNMT & 5254 & 457 & 19.34\\ 
 \hline
 \end{tabular}
\end{table}

\subsection{Nematus}

Nematus is an open source toolkit developed by Sennrich et al.(2017) from the Edinburgh NLP group, it is implemented in Python and based on the Theano framework (Theano Development Team, 2016) . They implemented an attentional encoder-decoder architecture which is similar to the one described by Bahadanau et al.(2015) , but there were several implementational differences.
\begin{itemize}
\item In the implementation of Nematus they initialized the decoder hidden state with the mean of the source annotation, rather than the annotation at the last position of the encoder backward RNN. 
\item They implemented a new conditional GRU with attention for Nematus.
\item In the decoder, they used a feedforward hidden layer with tanh non-linearity rather than a maxout before the softmax layer.
\item They didn’t implemented any additional biases for both encoder and decoder embedding layers.
\item The implementation of Bahadanau et al. (2015) used Look, Generate, Update decoder phases whereas in Nematus they implemented Look, Update, Generate which simplified decoder implementation significantly (See Table 1). 
\item They performed recurrent Bayesian dropout (Gal,,2015)
\item In Nematus they allowed multiple features (or “factors”) at each time step instead of a single word embedding at each source position, with the final embedding being the concatenation of the embeddings of each feature (Sennrich and Haddow, 2016).
\item They allowed tying of embedding matrices (Press and Wolf,2017;Inan et al.,2016).
\end{itemize}

\begin{table}[h!]
\centering
 \begin{tabular}{ |ll|ll|} 
 \hline
 \multicolumn{2}{|l|}{RNNSearch (Bahdanau et al., 2015)} & \multicolumn{2}{l|}{Nematus (DL4MT)}\\
  \hline  Phase & Output - Input & Phase & Output - Input \\
  \hline Look & $C_j$ $\leftarrow$ $S_{j-1}$ ,C & Look & $C_j$ $\leftarrow$ $S_{j-1}$, $Y_{j-1}$ ,C \\
  Generate & $Y_j$ $\leftarrow$ $S_{j-1}$, $Y_{j-1}$ ,$C_j$ & Update& $S_j$ $\leftarrow$ $S_{j-1}$, $Y_{j-1}$ ,$C_j$ \\
  Update & $S_j$ $\leftarrow$ $S_{j-1}$, $Y_{j}$ ,$C_j$ & Generate& $Y_j$ $\leftarrow$ $S_{j}$, $Y_{j-1}$ ,$C_j$ \\
  \hline
 \end{tabular}

\caption{Decoder phase differences between Nematus (DL4MT) and RNNSearch (Bahdanau et al., 2015)}
\end{table}

Given a source sequence ($x_1, . . . , x_T_x$ ) of length $T_x$ and a target sequence ($y_1, . . . , y_T_y$ ) of length $T_y$, let $h_i$ be the annotation of the source symbol at position $i$, obtained by concatenating the forward and backward encoder RNN hidden states, $h_i$ = [\overrightarrow{h_i} ; \overleftarrow{h_i} ], and $s_j$ be the decoder hidden state at position $j$.

\textbf{Decoder Initialization} Bahadanau et al. (2015) initialized the decoder hidden state $s$ with the last backward encoder state.
\begin{equation}
S_0 = tanh (W_{init}\overleftarrow{h_i})
\end{equation}

With $W_{init}$ as trained parameters. But in the Nematus implementation they used the average annotation instead:

\begin{equation}
S_0 = tanh (W_{init} \frac{\sum_{i=1}^{T_x}h_i}{T_x})
\end{equation}

\textbf{Conditional GRU with attention} Nematus implemented a novel conditional GRU with attention, $cGRU_{att}$. A $cGRU_{att}$ uses its previous hidden state $s_{j-1}$, the whole set of source annotations $C = {h_1, . . . , h_T_x}$ and the previously decoded symbol $y_{j-1}$ in order to update its hidden state $s_j$ , which is further used to decode symbol $y_j$ at position $j$.

\begin{equation}
s_j= cGRU_{att}(s_{j-1},y_{j-1},C)
\end{equation}


The conditional GRU layer with attention mechanism $cGRU_{att}$, consisted of three components : two GRU state transition blocks and an attention mechanism ATT in between. The first transition block, $GRU_1$, combines the previous decoded symbol  $y_{j-1}$ and previous hidden state $s_{j-1}$ in order to generate an intermediate representation $s_j^`$ with the following formulations:


\begin{align*}
s_j^` &= GRU_1(y_{j-1},s_{j-1})=(1-z_j^`)\bigodot \underline{s}_j^`+z_j^`\bigodot s_{j-1}^`,\\
\underline{s}_j^`&= tanh(W^`E[y_{j-1}]+r_j^`\bigodot(U^`s_{j-1})),\\
{r}_j^`&= \sigma (W_r^`E[y_{j-1}]+U_r^`s_{j-1}),\\
{z}_j^`&= \sigma (W_z^`E[y_{j-1}]+U_z^`s_{j-1}),
\end{align*}

, where $E$ is the target word embedding matrix, \underline{s}$_j^`$ is the proposal intermediate representation, $r_j^`$ and $z_j^`$ being the reset and update gate activation. In this formulation, $W^`$ , $U^`$ , $W_r^`$ , $U_r^`$ , $W_z^`$ , $U_z^`$ are trained model parameters; $\sigma$ is the logistic $sigmoid$ activation function.

In the Nematus implementation, the entire context set $C$ along with intermediate hidden state $s_j$ is taken as input in the attention mechanism ATT, in order to computer the context vector $c_j$ as follows:

\begin{align*}
c_j &= ATT(C,s_j^`)= \sum_{i}^{T_x}{a_{ij}}h_i\\
\alpha_{ij}&= \frac{exp(e_{ij})}{\sum_{k=1}^{T_x}exp(e_{kj})}\\
e_{ij} &= V_{\alpha}^T tanh(U_{\alpha}s_j^`+W__{\alpha}h_i)
\end{align*}
, where $\alpha_{ij}$ is the normalized alignment weight between source symbol at position $i$ and target symbol at position $j$ and $v_\alpha$, $U_\alpha$,$W_\alpha$ are the trained model parameters.

In the next step , $s_j$ is generated by the second transition block $GRU_2$ 

\begin{align*}
s_j^` = GRU_2(s_j^`,c_j)&=(1-z_j^`)\bigodot \underline{s}_j+z_j\bigodot s_{j}^`,\\
\underline{s}_j&= tanh(Wc_j+r_j\bigodot(Us_j^`)),\\
{r}_j&= \sigma (W_rc_j+U_rs_j^`),\\
{z}_j&= \sigma (W_zc_j+U_zs_j^`),
\end{align*}

, where \underline{s}$_j$ is the proposal hidden state, $r_j$ and $z_j$ being the reset and update gate activation. In this formulation, $W$ , $U$ , $W_r$ , $U_r$ , $W_z$ , $U_z$ are trained model parameters.

In Nematus, the two implemented GU blocks are not individually recurrent, they are only recurrent at the level of the whole cGRU layer. This way of combining RNN blocks is similar to what is referred in the literature as deep transition RNNs (Pascanu et al., 2014; Zilly et al., 2016) as opposed to the more common stacked RNNs (Schmidhuber, 1992; El Hihi and Bengio, 1995; Graves, 2013).

\textbf{deep output} Given $s_j , y_{j-1},$ and $c_j$ , the output probability $p(y_j|s_j , y_{j-1}, c_j )$ is computed by a softmax activation, using an intermediate representation $t_j$ . 

\begin{align*}
p(y_j|s_j , y_{j-1}, c_j ) = softmax (t_jW_o)\\
t_j=tanh (s_jW_{t1} + E[y_{j-1}]W_{t2} + c_jW_{t3}) \\
\end{align*}

,where $W_{t1},W_{t2},W_{t3},W_o$ are the trained model parameters.

They designed the Nematus to minimize the cross-entropy on parallel training corpus. For the training they used stochastic gradient descent, or one of its variants with adaptive learning rate (Adadelta (Zeiler, 2012), RmsProp (Tieleman and Hinton, 2012), Adam (Kingma and Ba, 2014)). Further, to optimize towards arbitrary, sentence level loss function they provided support for minimum risk training (MRT) (Shen et al., 2016). 

\begin{figure}
\includegraphics[width=\textwidth]{figures/nematus.png}
\caption{ Search graph visualisation for DE$\rightarrow$ EN
translation of "Hallo Welt!" with beam size 3 using Nematus} \label{fignm2}
\end{figure}

\subsection{Neural Monkey}
Neural Monkey is an open-source toolkit written using the TensorFlow machine learning library (Abadi et al., 2016). It provides a higher level API, such that it should be enough for the users to be familiar with the models on the equation level, without delving into implementation details. As compared to other open-source NMT toolkits, Neural Monkey provides a higher level of abstraction, along with a simple configuration mechanism that allows for fast prototyping and reusing trained models and experiment management.

The building blocks of Neural Monkey are not individual network layers like \textit{tfLearn} or \textit{Lasagne} , but they are more abstract objects like encoders and classifiers. These objects are parametrized so that their properties (e.g. number and sizes of hidden layers or dropout probability) can be set from a farther perspective. [Reference]. This design process allowed the authors to separate the configuration of the experiments from the actual code, which prevented the users from interleaving the configuration with other program logic. 

\begin{figure}
\includegraphics[width=\textwidth]{figures/nmonkey2.png}
\caption{ Creation of Dataset in Neural Monkey} \label{fignm2}
\end{figure}

In Neural Translation Systems, the loading and processing of datasets is one of the most important systems. The dataset in Neural Monkey is created in three step, first an input file is read using a Reader which load a file containing paths to JPEG images and load them as NumPy arrays, or read tokenized text as a list of lists of string tokens. The next phase is pre-processors, the series created by the readers is pre-processed by the system such as byte-pair encoding (Sennrich et al., 2016) which loads a list of merges and segments the text accordingly. Finally, the dataset-level pre-processors are applied of multiple series data to create the final dataset. 

\begin{figure}
\includegraphics[width=\textwidth]{figures/nmonkey1.png}
\caption{ Model workflow in Neural Monkey} \label{fignm1}
\end{figure}


The model in Neural Monkey is defined by different model parts such as encoders and decoders. The authors defined the encoders and decoders in more general than the classical Sequence to Sequence Learning. They defined the Encoders as parts of the model which take the input and compute a representation of it. They used runners for the execution of the Decoders. They designed different runners to represent different ways of running the model. They implemented trainer which is a runner to modify the parameters of the model, to collect the objective functions and use them in a optimizer. They used an object called TensorFlow manager to manage all the TensorFlow sessions. 

In order to validate the Neural Monkey’s performance the authors did a sanity check evaluation on the architechture introduced by Bahadanau et al. (2014) which became the standard baseline model in NMT research. 








The used a bi-directional GRU network with 500 hidden units in each direction as the encoder and for decoder they used an RNN decoder with 1000 units in the hidden layer. They used a simple ‘tanh’ projection instead of max-out projection as used in the original model. The original paper used Adadelta (Zeiler, 2012) optimizer, whereas for this comparative study they used Adam (Kingma and Ba, 2014) . The comparison of the models is shown in Table. 

\begin{table}[h!]
\centering
 \begin{tabular}{ |c|c|c| } 
  \hline Model & BLEU & epoch \\ 
  \hline  Bahdanau et al.(2014) - neural & 26.75 & 2.2\\
  Bahdanau et al.(2014) - neural & 28.45 & 6.0\\
  SMT & 33.30 & -\\
  \hline Neural Monkey - greedy decoding & 25.08 & 1.2\\
  Neural Monkey + CGRU – greedy decoding  & 27.35 & 1.6\\
  \hline
 \end{tabular}
\caption{Results achieved by Neural Monkey on the WMT14 News Task French to English dataset with the number epochs}
\end{table}

\section{Related Work}

  \chapter{Design and Methodology}
\section{Data Gathering}
\section{Neural Translation Model}
\section{Visual Interface}
  \chapter{Implementation}
In this chapter, the implementation details are talked about. The technologies that used are described, alongside the justification for selecting them. While the system has coherently isolated components, to be specific a subsystem of Data Gathering, Neural Translation Model and a Visualization Interface, each with different design strategies and implementation approach.

\begin{figure}[h]
\includegraphics[width=\textwidth]{figures/maindesign.png}
\caption{System Components} \label{system}
\end{figure}

\section{Data Gathering}
\subsection{Dataset for Fine Tuning}
\begin{figure}
\includegraphics[width=\textwidth]{figures/tedtalks.png}
\caption{A side-by-side figure showing TED Talk transcripts in English and Hindi} \label{fig1}
\end{figure}
\subsubsection{Beautiful Soup}
Beautiful Soup is one of the most widely used open source Python library designed by Leonard Richardson for quick turnaround projects like website scrapping. The decision to use Beautiful Soup for this research was motivated by the fact that It provides simple methods and Pythonic idioms for navigating, searching and modifying a parse tree. It’s a simple toolkit for dissecting the document and extracting the relevant information according to the user’s need. Beautiful Soup makes the handling of encodings much easier, it automatically converts incoming documents to Unicode and outgoing documents to UTF-8. The library sits on top of Popular parsers like lxml and html5lib which allows the user to try different parsing strategies or trade speed for flexibility. 
\subsubsection{Indic NLP Library}
Indic NLP Library is an open source Python Based Library for common text processing and Natural Language Processing in Indian Languages developed by Anoop Kunchukuttan.  The Indian Languages are originated from Sanskrit and are quite different from the Latin-Based Languages or other Asian Languages, so the other NLP Libraries doesn’t perform well on Indian Languages. Though the Indian languages share a lot of similarity among themselves in terms of script, phonology, language syntax, etc. There are very few researches going in the field of Hindi NLP, and Indic NLP Library is the only existing NLP library for the Hindi language. The library provides several functionalities such as Text Normalization, Tokenization, and Morphological Analysis.
\subsubsection{Moses Tokenizer}
Moses is one of the most successful implementations of the Statistical Machine Translation which was the dominant approach before the onset of Neural Machine Translation. Moses provides several tools for Statistical Machine Translation process, such as for Corpus preparation it provides tokenization, true casing and cleaning tools. As the process of preparation of corpus remains same for Neural Machine Translation Systems, the Moses Tokenizer was chosen here for the tokenization of the English text. The Moses tokenizer was chosen over other tokenizers as the Moses tokenizer is customized for corpus specific tokenization.


\subsubsection{TED Hindi English Corpus}
\begin{figure}
\includegraphics[width=\textwidth]{figures/traindataworkflow.png}
\caption{Workflow of the TED Talks Corpus Generator} \label{fig1}
\end{figure}
TED.com discontinued their Open API project and so there is no direct way to fetch the transcripts of talks. TED.com is a static website, so talks link location is static and can be accessed if prior information of talks details is obtained through Beautiful soup and $urllib$ library in python 

Firstly, the Beautiful Soup library is used to fetch the links of all the 411 TED Talks which are available in the Hindi language from the home page of the TED Official Website. The URL of the web page is passed to the method which implements the Beautiful Soup.  For each and every web page, the $find\_all()$ method finds all the “$a$” tags in the respective pages. Further, $attrs$ method is used to find all the $“href”$ which are having "$/talks$". The URL to the original talks is obtained and stored in a list for further parsing. The method is iterated through the entire list of pages having Hindi TED Talks. 

The transcripts for the talks are fetched from TED’s internal server which is having an open access and URL is appended "$transcript.json?language=en$" . Two different methods for English and Hindi are implemented to obtain the JSON from the respective URLs and load into the data file. 

The JSON is parsed by the respective methods and the transcript text is obtained. The JSON data consisted of time frames, translated text of available language which here Hindi and English respectively. The methods parse the JSON data, fetch the translation and write it in two separate text files “$Hindi.txt$” and “$English.txt$” to create the Hindi-English Parallel Corpus.

The text files generated by the methods were ill-formatted and contained junk texts. So, there was a need for Normalization and Tokenization of the text files. The text written in Hindi displays a lot of quirky behavior on varying input or multiple representations of the same character, so there was a need to canonicalize the representation of the text such that the NLP applications can handle the data inconsistent manner. The canonicalization of the text files handled issues such as,
\begin{itemize}
\item Non-Spacing characters like ZWJ/ZWNL
\item Multiple Representations of Nukta based Characters
\item Multiple Representations of two-part dependent vowel signs
\item Typing inconsistencies: e.g. use of pipe ($|$) for poorna virama 
\end{itemize}

Further the tokenizer in the library was used to tokenize the Hindi text and make it ready for corpora.

The Moses tokenizer comes with two perl scripts  which  is used for normalization and tokenization of the english side of the corpus.

The details of the corpus is showed in Table \ref{corpustable} 

\begin{table}[h!]
\centering
 \begin{tabular}{ |ccc| } 
  \hline Language & Number of Segments & Number of Tokens \\ 
  \hline  Hindi &  84157 & 0.2226\\
  English & 84157 & 1.0530\\
  \hline
 \end{tabular}
\caption{Details of the TED Talks Hindi-English Corpus}
\label{corpustable}
\end{table}

\subsection{Domain specific Test Data}
\subsubsection{Sumy}
Sumy is an Open Source python library and python command line utility for extracting a summary from HTML pages or plain texts, developed by Miso Belica. The package also contains a simple evaluation framework for text summaries.The decision to use sumy was driven by a number of factors.Sumy is an extractive text summarizer.Extractive text summarization techniques perform summarization by extracting portions of texts and constructing a summary, while the abstractive techniques like Google's TextSum learn the internal language representation to generate more human-like summaries,
and paraphrases the original text. Extractive text summarization makes more sense in the summarization of Blogs as it keeps the original text which highlights the author’s point of view and writing style. Secondly, the abstractive text summarization is technique is highly unfeasible considering the available infrastructure at this stage of research.Further, Sumy provides seven different summarization methods which gives a choice to choose the best summarization algorithm for the blog summarization. 

The methods are as follows: 

\begin{itemize}
    \item Luhn is one of the earliest suggested algorithms by the famous IBM researcher it was named after and it scores sentences based on the frequency of the most important words. The algorithm derives statistical information from word frequency and distribution to compute a relative measure of significance. The sentence which scores relatively higher than others is extracted to create a summary. 
    \item Edmundson is a heuristic method which implements previous statistic research of high-frequency words along with the three additional components: pragmatic words (cue words); title and heading words; and structural indicators (sentence location). 
    
\end{itemize}

\subsubsection{Summarizer}

\begin{figure}
\includegraphics[width=\textwidth]{figures/textsummary.png}
\caption{Figure depicting the summarization of Tripoto Blogs} \label{fig1}
\end{figure}
Tripoto is one of the biggest Travel Blogging Websites in the World. The method implements the Beautiful Soup library and fetches the list of 50 most popular blogs from the URL “$https://www.tripoto.com/trips$ “. The links to the respective blogs is written into a csv (comma separated values) file for the summarization.

The method which implements the sumy library reads the csv file and creates a 5-sentence summary for each and every blog. After experimenting with LexRank, TextRank and Latent Semantic Analysis (LSA), LSA is finally used as the summarizer based on the quality of the summary. The Summary is then saved in a csv file corresponding to their blog links, which is further used by the Visual Interface. The Google Translate API is used to generate the Hindi translation of the summaries which is used as the reference text for evaluation of the translation model. The final domain specific test data consists of 580 pairs of bilingual sentences in Hindi and English.

\section{Neural Translation Model}
\subsection{PyTorch}
PyTorch is an open source machine learning library for Python developed by \cite{paszke2017automatic}. PyTorch provides high level features like strong GPU acceleration and Dynamic Deep Neural Networks  built on a tape-based autograd system. The decision to use PyTorch was motivated by PyTorch implementation of OpenNMT(\citeauthor{opennmt},\citeyear{opennmt}) which is easily extensible and suitable for research. The decision to use OpenNMT-PyTorch for training the neural model has been discussed in Section \ref{sec:opennmt}.
\subsection{Seq2Seq Model with Attention}
The Seq2Seq model with Attention is well implemented by the OpenNMT Toolkit. The workflow of the English to Hindi Neural Translation Model is showed in Figure \ref{nmtwork}.

\begin{figure}
\includegraphics[width=\textwidth]{figures/nmtwork.png}
\caption{Workflow of the Neural Translation Model} 
\label{nmtwork}
\end{figure}

\subsubsection{Pre-Processing} 
Pre-processing is one of the most important component of the Neural Translation Model.The preprocess.py requires the training and validation data from source(english) and target(hindi) language.The validation text files are required by the system to evaluate the converge of training.The preprocess.py is configured according to the design requirements. The size of vocabulary for source and target languages is set to 50K which is well suited for a training data having 1.49 million bilingual sentence pairs.After running the preprocessing.py, the following files are generated by the OpenNMT system:
\begin{itemize}
    \item \textbf{demo.train.pt} A serialized PyTorch file which contains training data.
    \item \textbf{demo.valid.pt} A serialized PyTorch file which contains validation data.
    \item \textbf{demo.vocab.pt} A serialized PyTorch file which contains vocabulary data.
\end{itemize}

These indices are used by the OpenNMT-PyTorch system throughout the process of training and testing. 
\subsubsection{Training}
\label{opennmtmodels}
The Training sub-component consists of train.py which trains the model with the training data from the corpus. The default training configuration consists of 2-layer LSTM with 500 hidden units on both the encoder/decoder. The training component can be configured according to the design requirements. The default word embedding size of 500 for both source and target is maintained throughout the implementation. The training models were implemented using either Bidirectional RNN (\cite{45610}) or LSTMs(\cite{NIPS2014_5346}). \cite{DBLP:journals/corr/VaswaniSPUJGKP17} proposed the Transformer model which is based solely on attention-mechanisms and is different from the tradition Seq2Seq. But the Transformer model was not implemented in this research as the focus was to improve the translations only using LSTM. The following models were implemented using different configurations,
\begin{itemize}
    \item \textbf{OpenNMT-Model-1} Consisted of 4-layer LSTM on the Encoder side, and 4-layer LSTM on the Decoder side, RNN size of 500 units, Attention mechanism as Luong, SGD Optimizer with learning rate of 0.001 and Batch size of 64. 
    \item \textbf{OpenNMT-Model-2} Consisted of 4-layer LSTM on the Encoder side, and 4-layer LSTM on the Decoder side, RNN size of 500 units, Attention mechanism as Luong, Adam Optimizer with learning rate of 0.001 and Batch size of 64. 
    \item \textbf{OpenNMT-Model-3}  Consisted of 2-Layer Bidirectional-RNN on the Encoder side, and 2-layer LSTM on the Decoder side, RNN size of 500 hidden units, Attention mechanism as Luong, SGD Optimizer with learning rate of 0.1 and Batch size of 64. This implementation was motivated by the use of Bi-directional RNN to improve translation quality in Google's NMT (\cite{45610}).
    \item \textbf{OpenNMT-Model-4}  Consisted of 2-Layer Bidirectional-RNN on the Encoder side, and 2-layer LSTM on the Decoder side, RNN size of 500 hidden units, Attention mechanism as Luong, Adam Optimizer with learning rate of 0.001 and Batch size of 64. This implementation was influenced by \cite{DBLP:journals/corr/abs-1712-07628}'s research which concluded that Adam optimizer performs better than SGD in the initial phases of training.
    \item \textbf{OpenNMT-Model-5}  Consisted of 2-Layer Bidirectional-RNN on the Encoder side, and 2-layer LSTM on the Decoder side, RNN size of 1000 units, Attention mechanism as Luong, SGD Optimizer with learning rate of 0.1 and Batch size of 64.\cite{DBLP:journals/corr/BaroneHSHB17} showed that Deep RNNs obtain better translations though they reduce the training speed by significant margin, so the RNN size was increased for this Model.
    \item \textbf{OpenNMT-Model-6} Consisted of 2-layer Bidirectional-RNN on the Encoder side, and 2-layer LSTM on the Decoder side, RNN size of 1000 units, Attention mechanism as Luong, Adam Optimizer with learning rate of 0.001 and Batch size of 64. Its a re-implementation of OpenNMT-Model-5 with Adam optimizer motivated by the performance of Adam optimizer in OpenNMT-Model-4. 
    
\end{itemize}

These models were trained on a Nvidia K80 GPU for 100K steps.The OpenNMT system calculated the training accuracy and perplexity every 50 steps whereas the validation accuracy and perplexity was calculated every 10K steps. The translation models were saved every 5K steps for further re-training. 

\subsubsection{Translating}
The model generated in the previous step is used by translate.py to translate the source text to generate the pred.txt in the target language. The translate.py requires the src-test.txt and the trained model model-step$\_$100000.pt to  translate the source files to the target language. 

\subsection{Back Translation using Monolingual Corpus}
\label{backtrans}
\cite{DBLP:journals/corr/SennrichHB15a} suggested methods for improving the translation quality by making use of monolingual data for low-resource language sources. As discussed before, The IIT-Bombay Hindi Parallel Corpus contains around 1.49 million sentence pairs which is little less for creating optimal neural translation models. This problem can be addressed using the monolingual corpus.\cite{DBLP:journals/corr/SennrichHB15a} suggested two methods to improve the translation quality, first by using dummy source sentences and second, by synthetic source sentences. The results from their experiments showed that the first method is not very efficient and doesn't have much significance on the translation quality. The second method is called back-translation which involves automatic translation of
the monolingual target text into the source language. Then this synthetic source text is paired with the original monolingual target text and added to the human generated parallel training data. The results from their experiments showed that there was substantial improvement on WMT 15 English $\rightarrow$ German translation. 

\begin{itemize}
    \item \textbf{Create a back-translations} The main objective in this research is to create English to Hindi translations. For creating backs translations, a new Hindi to English training model is implemented using the same corpus. Then the monolingual sentences are translated using the model. The translations act as synthetic source language data.
    \item \textbf{Adding the translations to main corpus} The synthetic data and its equivalent human written monolingual data is integrated with the parallel corpus.
\end{itemize}

The IIT Bombay Hindi Monolingual Corpus which is mentioned in \ref{sec:mono} has around 45 million sentences. Creating Back-translations for such a huge monolingual corpus requires huge amount of GPU power which was not feasible for this research work. So, a fraction of the monolingual corpus, 50K sentences were used for creating back translations in English.Then the Hindi monolingual corpus and the English back-translations were integrated with the parallel data for re-training of the model. The detailed workflow of the process is depicted in Figure \ref{backtrans}.

\begin{figure}[h]
\includegraphics[width=\textwidth]{figures/backtrans.png}
\caption{Workflow of the Back Translation process} 
\label{backtrans}
\end{figure}

\subsection{Fine-Tuning with Domain Specific Training Data}
The process of fine-tuning involves the re-training of the pre-trained model with domain specific training data. The process is shown in Figure \ref{nmtwork1}. This is a complete re-iteration of the entire training process but with new domain specific data.
\begin{itemize}
    \item \textbf{preprocess.py} The source and target training data from the domain-specific TED Talks Hindi-English Corpus were configured along with validation text from the main IIT-Bombay Hindi-English Corpus.
    \item\textbf{train.py} For the domain specific training, the configuration of the OpenNMT-Model-6 was implemented.
    \item\textbf{translate.py} The translate.py takes the sentences in the source language and translates them into the target language.
\end{itemize}

\begin{figure}[h]
\includegraphics[width=\textwidth]{figures/nmtworkflow1.png}
\caption{Workflow of the fine-tuning process} 
\label{nmtwork1}
\end{figure}

\section{Visual Interface}
The Visual Interface constitutes a simple front end interface component, and is constructed using web application technologies - HTML, CSS, and Javascript to create the static website for visualization.

HTML and CSS are used in combination for mark up and style information. Javascript is used to access the Data Store which contains the translations of the summaries of the blogs.

\subsection{Visual Interface Landing Page}
\subsection{Blog Summary Display}
\section{Summary}
Chapter Four discussed about the various technology decisions made in creating the system. The chapter discussed in details the implementation of domain-specific corpus generation, domain-specific test data generation, neural translation model, and the visualization interface.
  \chapter{Evaluation}
\section{Data Gathering}
  \chapter{Conclusion}
\section{Data Gathering}

Random citation \cite{einstein} embed deed in text.



\bibliographystyle{agsm} % We choose the "plain" reference style
\bibliography{refs}                            %% to generate your bibliography.


\addcontentsline {toc}{chapter}{Appendices}       %% Force Appendices to appear in contents
\begin{appendix}
 \include{appendix1}
% \include{appendix2}
\end{appendix}


%\addcontentsline {toc}{chapter}{Bibliography}     %% Force Bibliography to appear in contents


\end{document}                                    %% END THE DOCUMENT




